\documentclass[11pt]{jsarticle}
\usepackage{setspace}
\usepackage{multicol}
\usepackage{latexsym}
\usepackage{mathrsfs}
\usepackage{url}
\usepackage{ascmac}
\usepackage[dvipdfmx]{graphicx}
\usepackage{theorem}
%\usepackage{marginfix}
\usepackage{framed}

\usepackage{fancyhdr}
\pagestyle{fancy}
\lhead{Binary Geohash}
\rhead{Binary Geohash}

%% for apple LaserWriter Series %%
%% 
\setlength{\topmargin}{-0.5in}
\setlength{\textwidth}{5.6in}
\setlength{\textheight}{8.8in}
\setlength{\oddsidemargin}{0.35in}
\setlength{\evensidemargin}{0in}

\setlength{\topmargin}{-0.5in}
\setlength{\textwidth}{5.6in}
\setlength{\textheight}{8.8in}
\setlength{\oddsidemargin}{0in}
%\setlength{\evensidemargin}{0in}
\setlength{\marginparwidth}{1in}

\usepackage{theorem}
\renewcommand{\baselinestretch}{1.1}
\setlength{\parskip}{0.25ex}
\renewcommand{\arraystretch}{0.95}

\title{GeoTree}
\author{}
\date{}

\begin{document}
\maketitle

\section{定義}

バイナリーコード $h \in\{0,1\}^*$ は GeoHash の二進表記で地球上の緯度,経度で指定される領域 $[-90, 90] \times [-180, 180)$ 内の矩形領域を精度 1 ビット単位で指定する.

\[
\left[ {\lfloor{i/2}\rfloor}*\frac{180}{2^{\lfloor{n/2}\rfloor}} -90, ({\lfloor{i/2}\rfloor}+1)*\frac{180}{2^{\lfloor{n/2}\rfloor}} -90 \right]
\times 
\]

集合ノード $v_S=(h, S)$ は,GeoHash のバイナリーコード $h$ と $h$ が定める領域 $[a_\mathrm{min}, a_\mathrm{max}) \times [g_\mathrm{min}, g_\mathrm{max}) \subseteq [-90, 90) \times [-180, 180)$ 内の座標点の集合 $S$ の組である.
分割ノード $v_d = (h, d, \mathit{left}, \mathit{right})$ は,以下の 4 つ組である.
\begin{enumerate}
\item
GeoHash のバイナリーコード $h$. 
\item
分割値 $d_h \in R$.

ただし,$h$ の精度(ビット長)を $l$ とすると,$l$ が偶数(0 を含む)ならば,$d_h$ は
$h$ が表す矩形領域の中央を東西に二等分する経線の経度であり,
$l$ が奇数であれば $d_h$ は $h$ の領域を南北に二等分する緯線の緯度である.
\item
左の子への参照,および右の子への参照.

ただし左の子および右の子は,それぞれ葉,内部ノード,もしくは空のいずれかである.
\end{enumerate}

\begin{defn}[GeoTree]
GeoTree は,集合ノードを葉とし,分割ノードを内部ノードしとする根付き二分木で,以下を満たすものをいう.
\begin{itemize}
\item[i)]
根ノード $r$ は長さ $0$ のバイナリーコード $h(r) = \varepsilon$ を持つ.
\item[ii)]
根以外のすべてのノード $v \neq r$ について,そのバイナリーコードの長さは親 $\mathit{parent}(v)$ のバイナリーコード 
$h(\mathit{parent}(v))$ の長さより 1 だけ長い.
\item[iii)]
左の子のバイナリーコードが表す領域は,その親のバイナリーコードが表す部分領域の第1分割領域,すなわち親の分割値によって生じる東または北の領域であり,
右の子のバイナリーコードが表す領域は,その親のバイナリーコードが表す部分領域の第2分割領域,すなわち西または南の領域である.
\end{itemize}
\end{defn}

子への参照が空である場合,そこには空集合を持つ集合ノード $v_\emptyset$ があると見なすことができる.

\end{document}
