\documentclass[11pt]{jsarticle}
\usepackage{setspace}
\usepackage{multicol}
\usepackage{latexsym}
\usepackage{mathrsfs}
\usepackage{url}
\usepackage{ascmac}
\usepackage[dvipdfmx]{graphicx}
\usepackage{theorem}
%\usepackage{marginfix}
\usepackage{framed}

\usepackage{fancyhdr}
\pagestyle{fancy}
\lhead{Binary Geohash}
\rhead{Binary Geohash}

%% for apple LaserWriter Series %%
%% 
\setlength{\topmargin}{-0.5in}
\setlength{\textwidth}{5.6in}
\setlength{\textheight}{8.8in}
\setlength{\oddsidemargin}{0.35in}
\setlength{\evensidemargin}{0in}

\setlength{\topmargin}{-0.5in}
\setlength{\textwidth}{5.6in}
\setlength{\textheight}{8.8in}
\setlength{\oddsidemargin}{0in}
%\setlength{\evensidemargin}{0in}
\setlength{\marginparwidth}{1in}

\usepackage{theorem}
\renewcommand{\baselinestretch}{1.1}
\setlength{\parskip}{0.25ex}
\renewcommand{\arraystretch}{0.95}

\title{GPS 軌跡の抽象化パターン}
\author{}
\date{}

\begin{document}
\maketitle

\section{準備}

地理点 $p = (y, x)$ は,緯度 $y \in (-90,90]$(北緯が $+$)と経度$x \in (-180, 180]$(東経が $+$)の組 $p \in (-90, 90] \times (-180, 180]$ で,地理的な位置を二次元座標上の点として表す.

地図グラフ $G=(V, E)$ は地理点の集合 $V$ とその二点を端点とする辺 $E \subseteq V^2$ の組で無向グラフとして表す.
辺 $(u,v) \in E$ の地理的長さ $\mathit{dist}(u,v) \in \mathbf{R}^+$ は,辺を道としたときの二つの地理点 $u, v$ の間の道のりである.

\section{パターン?}

GPS 軌跡の集合 $T=\{t_1, \ldots, t_n\}$ の各軌跡に対して,
地図グラフ $G=(V,E)$ 上の $t_i$ の近似辺集合 $E_i$ が得られたとする.

このとき,

\end{document}
